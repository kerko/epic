\chapter{XML}

\section{XML standard}
Our XML standard is really simple it uses mainly tag with attributes to create the different Entities and so on. Only entities which belong to other objects , like items a player carrys. These are processed as a reference tag inside the belonging tag.

Example Player with items inside:
\lstset{language=XML}
\begin{lstlisting}
<player name="Mike Moore" life="100" xcordinate="1" ycordinate="1">
        <item reference="knife" />
        <item reference="lamp" />
        <item reference="apple" />
    </player>
\end{lstlisting}
The items are specified as followed:
\lstset{language=XML}
\begin{lstlisting}
<equip-item
        id="knife"
        name="Knife"
        description="A sharp knife, use with caution!"
        damage="5" />

    <consum-item
        id="apple"
        name="Apple"
        description="A red apply. Seems to be tasty!"/>

    <collect-item
        id="lamp"
        name="Lamp"
        description="A bright lamp" />
\end{lstlisting}

Rooms are using attributes to specify their look and equally to player they have reference tags in them with contain creatures or items.
Example of a Room from the sample game:
\lstset{language=XML}
\begin{lstlisting}
<room id="dungeon" name="Dark Dungeon" width="3" height="3" message="You are alone. In front of you is a dark dungeon. Suddenly you hear some noise. RUN!">
        <door xcoordinate="2" ycoordinate="2" room="chamber" target="closeddoor" />
        <creature-ref reference="ghost" xcoordinate="2" ycoordinate="1" />
        <item-ref reference="apple" x="1" y="2" />
    </room>

\end{lstlisting}


        


\section{XML processing}